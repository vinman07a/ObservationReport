\documentclass[a4paper,titlepage,12pt]{turabian-researchpaper}

\usepackage[T1]{fontenc}
\usepackage[utf8]{inputenc}
\usepackage{lmodern}

\usepackage[english]{babel}
\usepackage{csquotes}

\usepackage{setspace}
\doublespacing

\usepackage{indentfirst}

\usepackage[notes,backend=biber]{biblatex-chicago}
\bibliography{sample}

\newgeometry{tmargin=1.5in, bmargin=1.5in, rmargin=1in, lmargin=1in}

\begin{document}
\title{A Report On Multiple Observations}
\author{Vincent Insinga}

\maketitle

\section{JOSHUA D. NELSON, Appellant v.
 SECRETARY, FL DOC et al., Appellees}


 The first Observation that I made in the process of writing this report was on
 Tuesday April fifth in the Eleventh Circuit Court of Appeals. The case that I
 observed was a death penalty case from Flordia. The appellant, Joshua D.
 Nelson, was appealing his conviction of a homocide that occured in 1995 to
 which he confessed.

 Even with the use of PACER (Public Access to Court Electronic Records), I was
 unable to access a brief that detailed the facts of the original case.
 I found a brief stating the nature of the appeal on PACER (this brief is
 attached), but not one
 relaying the nature of the charges originally brought against Joshua Nelson. I did however locate documents, including the brief and
 published opinion, relating to an appeal he made to the
 Supreme Court of Florida. I have cited these.

 \section{Facts of Case}

 In 1995, Joshua Nelson and Keith Brennan murdered Thomas Owens. They

 Nelson was
 eighteen years old and Brennan was too young to be sentenced to death.
 \autocite{scof}.

 \end{section}
 \section{Observations}
 I entered the building into a small room with three unique walls. The one that
 I faced immediately was one of bullet proof glass that seperated me from the
 security garuds inside; the wall to my right was filled with small lockers,
 into which I was instructed to place my phone and bottle of water, and to my
 left was the conveyor belt through which I was instructed to send my jacket,
 wallet, and shoes and the metal detector through which I myself had to step in
 order to enter the building.

 Upon entering, I was asked to sign a peice of paper with my name and the time
 at which I signed. The man sitting behind this counter questioned me about my
 age, me looking too young to carry a college I.D. After asking for directions
 to the room in which the arguments were being held, I continued up the
 elevator. The whole building had a feel of abandonment to it, being
 extraordinarily large yet almost entirely empty. Indeed none of the many large
 rooms were occupied, none short of this one. It is hard to describe the feel
 of those empty hallways, but it was a unique one. I was unsure weather to
 stand in awe of the buildings grandeur or in apathy of its smallness.

 The hearing was very short, much shorter than I expected. The hearing started
 at 2:00 PM and I signed out of the court at 2:40. There was a large digital
 clock on the desk of the clerk which indicated that the hearing could only
 last for a maximum of one hour and twelve minutes, allowing one half hour for
 each sides opening statements and six minutes for their rebuttals. Of course,
 neither of them did. Neither of them seemed particularly confident in the case
 itself going anywhere. In fact, I happened ti rude the same elevator as the
 prosecutor. I asked her how much time she spent preparing for this case and
 she replied "for this one in particular? I didn't really, because it's kind of
 an easy topic."
 \end{section}
 \section{Reflections}
 At one point, one of the judges asked the appellant lawyer ``do you know if
 Florida has allowed any executions to go through since the Hurst ruling?'' The
 lawyer replied that they had not. After that exchange I realised that I had
 not; I realised that there
 was a large likelihood that even if the court rejects his appeal, that Nelson,
 a man who has lived on death row for longer than I have been alive,
 will never face execution.
 \end{section}
 \section{Conclusions}
 Justice is slow. Justice is more than slow. Justice works at a tenth the speed
 any other thing. Truly,there is no absurdity which can do justice to the slowness of justice.
 Justice is, in fact slower than history. Legislation has come and gone, major
 decisions have been made and implemented, two and a half decades have passed since two
 teenagers killed a third with a box cutter and a baseball bat. Still, one of
 them does not yet know his fate. Indeed, it seems that the system waited
 itself out. In taking such a long time to move this case through the legal
 system, that very system has made the case obsolete.
 \end{section}
 \end{section}

\section{State vs Isaiah Simmons}

\section{Facts of Case}

The defendant, Isaiah Demarko Simmons was dating two women simultaneously,
whithout either of them having knowledge of this. Melanie Davis is in her late
twenties and Nicole Simpson is inher forties. Melanie Davis was aware of
Simpson's existence, but only as a kind mentor helping her boyfriend get a job.

Soon, each found out about the other and, as could be expected, problems ensued
for the relationships. Though, Davis would later state that she and Simmons did
not stop dating, they "were just having the problems you would have dating any
white boy" (both of the two are African-American). Simpson however, decided to
terminate her relationship with Simmons. Strangely, it was not the two women
who got angry - it was Simmons.

Isaiah Simmons sent a long series of threatening text messages to Melanie Davis
before she eventually stopped answering. He then began to threaten to come to
her home. Unfortunately, these threats were not as empty as those sent prior.

Davis arrived home to her door on the floor, instead of proceeding inside, she
ran back out and called the police. Additionally, she showed the police the
text messages sent to her by Simmons and they called the number, asking the
person who answered to verify their identity. Isaiah Simmons replied that that
was indeed who he was. Upon entering the home, they discovered the massive
amount of damage that had been done. Wires were cut; taps were opened all the
way, so as to cause water damage; tires were slashed; things
were broken and smashed. While Melanie Davis still claimed in court that she
was unsure whether the perpetrator was indeed Simmons, It is clear that whoever
broke into her home that day unleashed all of their anger and rage on that
house.

Soon after this, Nicole Simpson reported her car stolen and house stolen from
(Simpson had a key, having lived there at a point). When the police arrived,
she insisted that no charges be filed and that her car not be reported stolen,
sure that she could call Simmons. Soon afterwords she arranged with Simmons to
meet him somewhere to get her car back. The police did not head this request and
apprehended Simmons while he was travelling to that location on which he agreed
to meet Simpson. Upon searching the car, the police found marijuana and added
a violation of drug laws to his charges. Simmons was arrested and taken into
custody.

While in jail, he spoke on the phone often with both women, especially Melanie.
During these conversations, they both said that they forgave him and tried to
help him with his legal issues.

\end{section}

\section{Observations}
	This case challenged, not only my preconceived expectations about court, but
	also those notions which were set forward by the first case that I
	observed. This building, despite belonging to a lower court, had a much
	grander feel. These buildings are about the same size,  but
	nonetheless, the small room through which I entered the Elbert P.
	Tuttle court house, enclosed by bulletproof glass did not feel the same
	as the large, modern entryway that led into the Cobb County Superior
	courthouse. The courtroom itself made a similar distinction, instead of
	the hushed juryless room, above which three justices sat, in the
	eleventh circuit, the courtroom in the superior court was much more
	formal and inviting.

	The security measures taken here were far less strict. In my first
	observation, I entered the building only to see two people looking at
	me through bulletproof glass, separated by two sets of doors (or one
	metal detector and a door) in either direction. I was instructed then
	to place my water bottle and phone in a locker and pass my shoes,
	jacket, and wallet through a conveyor belt, before stepping through a
	metal detector. The process that I underwent
	upon entering Superior Court bore no resemblance to this. The entrance,
	with its high ceilings and elegant tiled floors, was open and
	welcoming. Again, I was still required to put my wallet, shoes, phone, and
	jacket, through a conveyor belt and scanner, but the process was far
	more relaxed.

 The judge began the session by reading a long speech to the jurors, detailing
proper rules and procedures.

The Prosecuter started the trial with a serious tone "RESPECT", she announced
in a bellowing voice, "This trial is about respect, or, more correctly, Isaiah
Simmons' lack of respect for property, for law, for women..." She continued to
lay down the fact pattern of the case in a very engaging anduhjuh9ihu manner.
Her voice would get quiet as she would slowly, gradually, and intentionally
begin to whisper before, once again, raising her voice. "So, you know, they
break up... and what does he do?" she posed this question in a voice so faint
that i could hardly hear, before abruptly answering it with "HE GETS ANRGY".
This is a tactic that she employed repetitively, quietly explaining the
situation before again exclaiming "HE GETS ANGRY" and detailing another
offense.

The defendant lawyer was all too quick to dismantle the air of seriousness and
overwhelming guilt created by the Prosecutor. He began with a statement that
absolutely obliterated the idea of formality in the court. "Do you know what I
love about trials?" he asked in a light tone, "They're like pancakes!" No, I
assure you, this is no a fabrication; it is not an accedemic dishonesty, in
fact, either of the two students with whom I attended this case can indeed
confirm that the attorney, Jason Treadaway, began his arguments in this trial by
comparing trials with a sweet breakfast food. He continued, saying "No matter
how you make 'em, or how thin you can get 'em in a non stick pan, they've
always got two sides."

The entirety of the beginning arguments played out as a battle between both
sides. The burden, of course, lied on the shoulders of the defendant to undo
all which the prosecutor had previously done. The prosecutor previously
said "As you can tell, there's going to be a lot of drama. It's going to be
like day time television 'Oooh, who's with who? Who's doing what?' Just
nonsense drama; it doesn't matter!" before urging the jury to ignore all of the
drama and focus on those charges that needed to be proved:
\begin{itemize}
  \item Burglary in the First Degree (two counts)
  \item Theft By Taking
  \item Criminal Damage to Property in the Second Degree
  \item Terroristic Threats
  \item Violation Of Georgia Controlled Substances Act
\end{itemize}
Instead, the defendant lawyer dismissed that idea, saying that the drama
was very important to understand the nature of the case. He argued that, while
"you cannot convict a man on the basis of him being among the top ten least
stellar boyfreinds of all time, which he might well have been" that
understanding that would be a vital part of this case.
It is worth noting that these six charges were broken up into two different
cases, one for Isaiah Simmons' transgressions against Nicole Simpson (burglary
in the first degree, theft by taking, and violation of Georgia Controlled
Substances Act) and
another for those involving Melanie Davis (burglary in the first degree,
criminal damage to property in the second degree, terroristic threats).

This trial presented some excellent examples of the effect that tone and body
language have on the way speech is perceived. Both sides, the defense and the
prosecution, used the same exact  quote from Nicole Simpson. "He is just being
childish" - the prosecuter said this in a small, meak voice, so as to portray
the image of a starled and helpless woman speeking to police. The defendant
lawyer spoke that sentence confidently between chuckles. 

There were several objections during the defendant's argument, causing
complications, in this case in particular, becuase of a request from the
defendant. The defendant asked that the screens should all be blanked and the
jury evacuated any time an objection was made. On the third or the fourth time
that this occured, the judge was visibly annoyed and frustrated; he finally
made himself heard to the defence. He was very clear in saying that "it's your
right" and "it's your decision". Nonetheless, his frustration leaked into his
voice.
\end{section}
\end{section}
\end{document}
